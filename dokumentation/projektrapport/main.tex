\documentclass[a4paper]{article}

\usepackage[swedish]{babel}
\usepackage[T1]{fontenc}
\usepackage[utf8]{inputenc}
\usepackage{graphicx}
\usepackage{float}
\usepackage{comment}
\usepackage{appendix}
\usepackage[backend=biber, style=ieee, sorting=none]{biblatex}
\usepackage[hidelinks]{hyperref}

\floatstyle{plaintop}
\restylefloat{table}

%\DeclareFieldFormat{url}{\bibstring{urlfrom}\addcolon\space\url{#1}}
%\addbibresource{referenser.bib}

\newcommand\namn{Larmsystem}
\newcommand{\todo}[1]{\marginpar{TODO: #1}\vspace{1cm}}
%\newcommand{\comment}[1]{\marginpar{Comment: #1}}

\newcommand{\subsubsubsection}[1]{\paragraph{#1}\mbox{}\\}
\setcounter{secnumdepth}{4}
\setcounter{tocdepth}{4}

% Nr, författare, titel, år, url, hämtad
\newcommand{\citewebpage}[6]{\noindent [#1] #2, "#3," #4. [Online]. Tillgänglig: \url{#5},  hämtad #6\\}

\restylefloat{table}

\begin{document}

\thispagestyle{empty}

\begin{center}
    \parskip=14pt
    \vspace*{3\parskip}

    {\LARGE Utveckling av modulärt larmsystem för mindre lokaler}

    {\large Datateknik - Datatekniskt projekt, grupp 11

    Titus Blosse, Viktor Frideen, Nazif Kadiroglu, Markus Moen, Lukas Schiavone, Fredrik Österström

    \today}

    \rule{7cm}{0.4pt}\\
\end{center}
\newpage

\thispagestyle{empty}

\tableofcontents
\newpage

\thispagestyle{empty}

\section*{Ordlista}

\begin{description}
    \item[CAN:] Controller Area Network
    \item[IDE:] Integrated Development Environment
    \item[MD407:] Mikrodator av typen MD407
    \item[USART:] Universal Synchronous and Asynchronous Receiver-Transmitter
    \item[DICP:] Dynamic Identification Configuration Protocol
\end{description}
\newpage

\pagenumbering{arabic}

\section{Inledning}
Inbrott är ett vanligt problem i Sverige.
År 2019 anmäldes 75.250 inbrottsstölder.
En minskning med 14 \% från året innan [1].
Vanliga sätt att förhindra inbrott är grannsamverkan, certifierade lås mot inbrott och hemlarm.
Larmsystem minskar risken för ett inbrott och möjliggör ytterligare minskning både i Sverige och i resten av världen.

\subsection{Syfte}
Ett modulärt larmsystem som informerar dess användare om oönskade intrång ska utvecklas.
Systemet ska lämpa sig för säkring av mindre byggnader som kontor och lägenheter.
Möjlighet ska finnas för systemets användare att anpassa systemet efter sina förutsättningar.

\subsection{Mål}

Produkten ska erbjuda ett flertal larmkomponenter som kan anslutas till en larmcentral.
Varje enhet ska kunna konfigureras och anpassas för att passa kundens specifika situation.
Till exempel ska kunden kunna justera antalet anslutna larmkomponenter eller på vilka villkor komponenterna alarmerar.
Dessutom ska systemet vara dokumenterat och fackmannamässigt utfört med förutsättningar för expansion.

Grunden till systemet som ska utvecklas kommer vara en centralenhet till vilken användaren kan ansluta periferienheter. De övergripande målen med centralenheten är att användaren ska kunna konfigurera periferienheterna genom den och att den ska kunna larma när ett alarm sätts på en periferienhet.

Två periferienheter kommer att utvecklas, ett dörrlarm och ett rörelselarm.
Dörrlarmets ska kunna starta ett larm då en dörr kopplad till en sensor öppnas.
Rörelselarmet ska använda två sensorer, en vibrationssensor och en avståndssensor och kommer larma vid upptäckta störningar.
De olika sensorernas toleranser ska kunna bestämmas på centralenheten.
Ifall en tolerans har bestämts ska ett lokalt larm först sättas hos periferienheten, ett alarm skickas till centralenheten om toleransen överskrids.

Att ett alarm har gått ska kommuniceras till användaren via en terminal på centralenheten och röda lampor.
När ett larm startar på centralenheten ska det krävas en manuell inmatning av fyrsiffrig kod på en ansluten knappsats för att larma av.
Denna kod är förutbestämd på systemet men ska kunna ändras av användaren.

Utöver basfunktionaliteten kommer även extra funktionalitet implementeras.
Denna extra funktionalitet kommer innebära ett självlärande system, ett testläge för periferienheterna, ett lokalt larm för rörelsesensorn och ett inbyggt skydd mot återuppspelningsattacker, en attack där förövaren återger tidigare skickad data i dataintrångsyfte.

Larmsystemet kommer utformas för att passa till mindre lokaler som exempelvis kontor, lägenheter och villor på grund av hårdvarubegräsningar.

\subsection{Projektdrift}
Mycket av all utveckling har skett på distans i och med dom restriktioner som legat på sociala sammankomster under utvecklingstiden.
Stor del av all kommunikation kring projektet har därför skett via kommunikationsverktygen Zoom, Discord och Messenger.

Majoriteten av testen utfördes direkt på hårdvaran i Chalmers lokaler.
Vissa tester har genomförts med hjälp av mjukvaran Simserver som tillåter simulering av MD407-kortet.
Funktionaliteten hos Simserver är begränsad och därför har den största delen av testerna behövts utföras i Chalmers laborationssalar.

Mjukvaran för projektet är skriven i programmeringsspråket C, ett språk som lämpar sig väl för utveckling mot mikrodatorn MD407 vilket är den hårdvara som driver systemet.

Versionshantering har gjorts med hjälp av git.
Ett gemensamt fjärrarkiv skapades på GitHub där all kod för projektet samlades.

\newpage

\section{Teknisk beskrivning}
I detta avsnitt ges en fördjupad inblick i hur systemet hänger ihop och fungerar.
Den mjuk- och hårdvara som projektet baseras på beskrivs även i mer detalj.

\subsection{Teknisk bakgrund}
Hårdvaran som systemet körs på är en MD407-mikrodator.
Mjukvaran är skriven i programmeringsspråket C.
MD407 har begränsat med minne vilket gör att programmen som skrivs för mikrodatorn bör vara små och effektiva.
C är ett lågnivå-programmeringsspråk vilket medför att programmeraren har stor kontroll över minneshanteringen i programmet.
C lämpar sig alltså för utveckling mot MD407.

Systemet kretsar kring att flera MD407 sammankopplas, en av dessa mikrodatorer agerar som centralenhet och att resterande används för att ansluta periferienheter.
MD407-korten kommunicerar via ett protokoll som baserats på CAN (Controller Area Network), ett protokoll för kommunikation mellan datorer.
% Ett kommunikationsprotokoll har sedan byggts ovanpå CAN som är anpassat för systemet.

\subsection{Systemöversikt}

\begin{figure}[H]
    \centering
    \includegraphics[width=0.7\textwidth]{canbuss-pp.png}
    \caption{Översikt över systemet}
    \label{fig:oversikt}
\end{figure}

Ett MD407-kort väljs som centralenhet för systemet.
Från denna enhet körs huvudprogrammet.
Centralenheten sköter huvudparten av systemets funktion.
Centralenheten registrerar vilka periferienheter som är anslutna, kontrollerar kontinuerligt så att inget alarm gått hos periferienheterna och läser inmatning från en ansluten dator för att styra systemet.
Enheterna är sammankopplade via CAN-bussen som i Figur \ref{fig:oversikt}.

Två typer av periferienheter finns tillgängliga för anslutning: en dörrenhet och en rörelseenhet.
Dörrenheten kontrollerar huruvida en dörr är öppen eller stängd. Rörelseenheten mäter avståndet till föremålet framför den och registrerar om detta förändras med en av sina sensorer och mäter vibrationer med sin andra.
Om en dörr är öppen längre än bestämd tid eller om rörelsesensorn registrerar ett avstånd mindre än sitt gränsvärde kommer periferienheterna larma.
Användaren har möjlighet att bestämma tiden för enskilda dörrar och gränsvärdet för avståndssensorn via USART på centralenheten.
%Om larmet fortsätter tillräcklig länge kommer periferienheten i fråga skicka ett meddelande till centralenheten.

\subsubsection{Kommunikationsprotokoll}
Ett protokoll har utvecklats för kommunikation mellan enheterna.
Basen till protokollet är CAN-meddelanden(Controller Area Network) som skickas mellan enheterna.

\begin{table}[H]
    \centering
    \caption{CAN-meddelandets uppbyggnad och ett exempel på hur ett alarm-meddelande från en periferienhet med ID ett ser ut.}
    \label{tab:meddelandestruktur}

    \resizebox{\textwidth}{!}{\begin{tabular}{|c||c|c|c|c|c|}\hline
        Fältnamn & Riktning & Meddelande-ID & Enhets-ID & Längd & Data \\\hline
        Antal bitar & 1 & 6 & 4 & 3 & 64 \\\hline
        Exempel & TO\_CENTRAL & ALARM & 1 & 0 & 0 \\\hline
    \end{tabular}}
\end{table}

%\begin{table}[H]
%    \centering
%    \caption{CAN-meddelandets uppbyggnad och ett exempel på hur ett alarm-meddelande från en periferienhet kan se ut.}
%    \label{tab:meddelandestruktur}

%    \resizebox{\textwidth}{!}{\begin{tabular}{|c|c|c|c|c|c|c|c|c|c|c|c|c|c|c|}\hline
%        \multicolumn{1}{|c|}{Riktning} & \multicolumn{6}{|c|}{Meddelande-ID} & \multicolumn{4}{|c|}{Enhets-ID} & \multicolumn{3}{|c|}{Längd} & \multicolumn{1}{|c|}{Data (8 bytes)} \\\hline
%        0 & 0 & 0 & 0 & 0 & 0 & 0 & 0 & 0 & 0 & 1 & 0 & 0 & 0 & 0 \\\hline
%        \multicolumn{1}{|c|}{TO\_CENTRAL} & \multicolumn{6}{|c|}{ALARM} & \multicolumn{4}{|c|}{NodeId} & \multicolumn{3}{|c|}{0} & \multicolumn{1}{|c|}{0} \\\hline
%    \end{tabular}}
%\end{table}

\begin{table}[H]
    \centering
    \caption{Samtliga meddelandetyper och deras prioritet.}
    \label{tab:meddelandetyper}
    \begin{tabular}{|c||c|}\hline
        Prioritet & Meddelande-ID \\\hline\hline
        0 & ALARM \\\hline
        1 & ERROR \\\hline
        2 & ALARM\_OFF \\\hline
        3 & POLL\_REQUEST \\\hline
        4 & POLL\_RESPONSE \\\hline
        5 & DICP\_REQUEST \\\hline
        6 & DICP\_RESPONSE \\\hline
        7 & TOL\_SET \\\hline
        8 & ACTIVE\_ON \\\hline
        9 & ACTIVE\_OFF \\\hline
    \end{tabular}
\end{table}


\noindent CAN är ett protokoll för kommunikation och hantering av meddelanden mellan enheter.
CAN-protokollet måste därför finnas hos både periferienhet och centralenhet.
Koden är därför strukturerad så att samma kod kan köras på valfri enhet oavsett typ.
Med hjälp av riktning och enhets-ID (enhets-identifikation) kan alla enheter ta emot och skicka meddelanden till godtycklig mottagare.
Varje CAN-meddelande byggs upp av fem fält som visat i Tabell \ref{tab:meddelandestruktur}:

\begin{description}
    \item[Riktning:] Detta fält anger om meddelandet går till eller från centralenheten.
    Fältet sätts till TO\_PERIPHERAL, motsvarande noll om meddelandet går från centralenhet till periferienhet och TO\_CENTRAL, motsvarande ett om meddelandet går från periferienhet till centralenhet.
    Detta fält tillåter en minskning av antalet meddelandetyper då en generell meddelandetyp som ALARM kan användas åt båda hållen.

    \item[Meddelande-ID:] Meddelanden delas upp i meddelandetyper, till exempel: ALARM och ERROR.
    Alla meddelandetyper finns samlade i en uppräkningstyp.
    Ordningen i uppräkningstypen bestämmer meddelandets prioritet.
    Ju närmare noll, desto högre prioritet.
    Samtliga meddelandetyper finns samlade i Tabell \ref{tab:meddelandetyper}.

    \item[Enhets-ID:] Varje ansluten enhet tilldelas ett unikt enhets-ID, mellan 0 och 14.
    15 är reserverat för ID-tilldelning. Ett lägre enhets-ID innebär högre prioritet.
    Centralenheten tilldelar enheterna ett ID i ordningen de anslöt sig i.

    \item[Längd:] Längd är en indikation på hur mycket data som skickas med i meddelandets datafält.
    Många meddelanden skickas med längd noll då mycket information om meddelandet kan utläsas från meddelande-ID och enhets-ID.

    \item[Data:] I detta fält lagras den data som skickas med i meddelandet.
    Ett meddelande kan lagra upp till 64 bitar av data.
\end{description}

Då ett meddelande ska skickas anropas en avsändarfunktion hos den enhet som vill skicka.
För varje meddelandetyp finns en unik avsändarfunktion som skapar ett CAN-meddelande med de nödvändiga parametrarna.
Funktionen lägger sedan ut meddelandet på bussen så att mottagaren kan plocka upp det.

För centralenhet och periferienhet finns unika mottagningsfunktioner.
Mottagningsfunktionen undersöker riktnings- och enhets-ID-fälten hos mottaget CAN-meddelande.
Funktionen avgör sedan huruvida mottaget meddelande berör enheten.
Om så är fallet kontrolleras meddelandetypen och en funktion för att hantera meddelandet kallas.
För varje meddelandetyp finns en unik hanteringsfunktion som utför de instruktioner meddelandet medförde.

\subsubsection{Centralenhet}
\begin{figure}[H]
    \centering
    \includegraphics[width=0.5\textwidth]{central-oversikt-pp.png}
    \caption{Översikt över centralenheten och dess komponenter}
    \label{fig:oversiktcentral}
\end{figure}

Centralenheten sköter huvuddelen av systemets funktionalitet.
En del av funktionaliteten är att kunna alarmera, avalarmera och konfigurera systemet.
För att kunna göra detta är ett larmdon, en knappsats och en USART-terminal kopplade till centralenheten enligt Figur \ref{fig:oversiktcentral}.
Enheten ansvarar över att övervaka alla anslutna enheter och alarmera om ett alarm uppstår hos någon av dem.
Om alarm uppstår kan det avaktiveras genom att mata in en fyrsiffrig kod på ansluten knappsats.

Centralenheten både lyssnar efter meddelanden och skickar meddelanden till periferienheterna kontinuerligt.
Ett meddelande till centralenheten kan till exempel innefatta ett alarm, ny ID-förfrågan eller att det uppstått ett fel.

Centralenheten skickar kontinuerligt förfrågningar, även kallat pollingmeddelande, till periferienheterna.
Detta för att undersöka så att inga enheter kopplas bort samt för att förebygga återupspelningsattacker.
I varje pollingmeddelande skickas en tidsstämpel från centralenhetens drifttid.
Enheten som mottager pollingmeddelandet förväntas svara centralenheten inom ett visst tidsintervall.
I pollingsvaret ska tidsstämpeln återfinnas fast inverterad.
När centralenheten mottager ett pollingsvar inom tidsramen med en korrekt inverterad tidstämpel skickas ett nytt pollingmeddelande till nästa anslutna enhet.
Om korrekt pollingsvar ej mottages startas ett alarm från centralenheten.

De instruktioner som matas in av användare via ansluten konsol (USART i Figur \ref{fig:oversiktcentral}) hanteras av centralenheten.
För att inte störa programflödet kontrolleras inmatningen av tecken periodiskt istället för att systemet väntar på inmatning från användare.
Inmatade tecken sparas i en buffer.
Då buffern är full eller en blankrad matas in av användare tolkas tecknen i buffern.
Om ett korrekt kommando har matats in kallas en funktion som hanterar kommandot.

%skickas för att säkerställa att enheten i fråga fortfarande är ansluten och inte larmar. I pollingmeddelandet

\subsubsection{Periferienhet}
\begin{figure}[H]
    \centering
    \includegraphics[width=\textwidth]{periferi-oversikt-pp.png}
    \caption{Översikt över periferienheterna och dess komponenter}
    \label{fig:oversiktperiferi}
\end{figure}

När periferienheten initialiseras hämtas en identifierare från centralenheten via CAN.
Detta görs genom att anta en standard-identifierare 15 och att skicka ett DICP(Dynamic Identification Configuration protocol)-meddelande.
Centralenheten tilldelar periferienheten en identifikator och besvarar DICP-meddelandet med den tilldelade identifikationen.
Vid framtida kommunikation använder periferienheten sig av sin unika identifierare för att skicka meddelanden.

För att inte centralenheten ska initiera ett alarm måste periferienheterna kunna svara på dess förfrågan.
Vid en förfrågan skickar centralenheten ett meddelande till en uppkopplad periferienhet.
I meddelandet anger centralenheten den lokala tiden då förfrågningen skickades.
Periferienhetens uppgift är att bitvis invertera tiden innan den besvarar meddelandet.
Centralenheten verifierar att den inverterade tiden stämmer överens med tiden då den skickade meddelandet.
Om en avvikelse i tiden uppstår så larmar centralenheten med ett alarm.

Periferienheterna avlyssnar sina sensorer och rapporterar ett alarm till centralenheten om tröskeln för sensortypen är överskriden.
Det finns tre typer av sensorer, en dörr-, en avstånds- och en vibrationssensor som är kopplade till respektive periferienhet enligt Figur \ref{fig:oversiktperiferi}.
Larmdonet i figuren används för att alarmera lokalt och att visa för användaren att ett alarm har uppstått.

\begin{description}
    \item[Dörrsensor:] Dörrsensorns uppgift är att upptäcka om en dörr är öppen eller stängd.
    Värdet på sensorerna läses av från PE0, PE2, PE4, PE6, PE8, PE10 och PE12, portpinnar på MD407 som är konfigurerade med Pull-up-resistorer.
    Vid öppnad dörr bryts kretsen och det logiska värdet blir ett.
    Vid sluten krets är det logiska värdet på kretsen noll.

    \item[Avståndssensor:] Avståndssensorns uppgift är att upptäcka skillnader i avstånd.
    För att kunna läsa ett värde från sensorn måste en logisk etta skrivas till trig-pinnen i minst 10 \textmu s.
    Sensorn svarar med att skriva en logisk etta på echo-pinnen.
    Längden av denna logiska etta motsvarar det uppmätta avståndet [3].

    \item[Vibrationssensor:] Vibrationssensors uppgift är att upptäcka vibrationer.
    En vibrationssensor används ofta för att mäta böjningar, beröring, vibration eller stötar.
    Sensorn kan rapportera en vibration både analogt via A0-pinnen eller digitalt via D0-pinnen på sensorn.
    D0 används för att läsa det logiska värdet på sensorn.
    Noll motsvarar en vibration och ett motsvarar standardtillståndet [4].
\end{description}

Dörrsensors tröskel justeras genom att skapa ett tidsintervall som kan konfigureras i Centralenheten utifrån kundens begäran. 
När en dörr öppnas kommer ett lokalt alarm triggas igång. 
Om dörren är öppen längre än det angivna tidsintervallet kommer även Centralenheten få ett alarm. 

Avståndssensorn samlar kontinuerligt in mätvärden för avståndet.
Om avståndet är mindre än den inställda tröskeln skickar enheten ett alarm till centralenheten.
Är avståndet däremot mindre är två gånger tröskeln kommer sensorn att alarmera lokalt.

Vibrationssensorns känslighet justeras direkt på sensorn.
Detta görs genom att ställa in en potentiometer med hjälp av en skruvmejsel.
Sensorn alarmerar lokalt och till centralenheten direkt vid en upptäckt vibration.

\newpage

\section{Metod}
Detta kapitel beskriver hur projektet har utförts; vad som har gjorts och hur det har gjorts.
Varje delsystem gås igenom för sig men de olika komponenterna har ofta utvecklats parallellt då det finns många delar av dem som överlappar.
Att delsystemen är kompatibla och kan samarbeta med varandra är viktigt för att systemet i sin helhet ska fungera.

\subsection{Inledande arbete}

Det första steget i utvecklingen var att ta fram en överblick på hur systemet skulle komma att se ut.
Alltså att ta fram den generella designen av systemet och fastställa vilka komponenter som skulle utgöra systemet.
Systemet delades upp i tre delar:
periferienheter, centralenhet samt kommunikationsprotokoll.
Att dela upp systemet på detta vis föreföll naturligt då utgångspunkten för projektet var att ha en centralenhet som ett flertal sensorer kunde anslutas till.
För att systemet skulle kunna hantera ett stort antal sensorer var det uppenbart att ett robust kommunikationsprotokoll behövde utvecklas.
Ytterligare en anledning till att dessa tre delar valdes var för att de har tydliga avgränsningar sinsemellan.
Dessa avgränsningar skulle komma att underlätta vid uppdelning av arbetsbördan.
Därefter delades arbetsgruppen upp i tre undergrupper.
Varje undergrupp fick ansvar för en del av systemet.
I början av projektet jobbade varje undergrupp enskilt på de olika delarna.
Senare i projektet när systemet närmade sig färdigställning återgick gruppen till att arbeta tillsammans.

\subsection{Utveckling av kommunikationsprotokoll}
Arbetet med CAN-protokollet påbörjades teoretiskt med att hitta en standard för hur meddelanden skulle vara strukturerade.
För att komma fram till en meddelandestruktur skapades en lista på de grundläggande meddelandetyper som skulle användas.
Utifrån listan av meddelandetyper kunde sedan meddelandestrukturen utformas.
När en standard för meddelandestruktur var definierad påbörjades arbetet med att realisera de olika meddelandetyperna.
Alla typer av meddelanden, dess funktion och dess parametrar dokumenterades och gjordes lättillgängligt i kodbasen.
I samband med framtagningen av meddelandetyperna och meddelandestrukturen lades grunden för det pollingsystem som skulle implementeras i centralenheten.
Att centralenheten skulle centrera runt ett pollingsystem påverkade designen av meddelandestrukturen.
Pollingsystemet skulle innebära att det kontinuerligt skulle ligga mycket meddelanden på bussen.
För att minska trafiken på bussen togs beslutet att inte använda sig av bekräftelser för mottaget meddelande.

Då funktionalitet för att skicka meddelanden skulle tas fram togs beslutet att ge varje meddelandetyp en egen funktion som skapar och skickar ett meddelande av den specifika typen.
Lösning upplevdes som effektivare än att varje meddelande skulle behöva gå genom ett flertal funktioner för att skickas, vilket var den ursprungliga designen.
Att göra på detta vis minimerade även mängden kod.
Då mer funktionalitet implementerades, som för att till exempel uppdatera tolerans eller sätta antalet dörrar, togs fler meddelandespecifika funktioner fram för att stödja den nya funktionaliteten.

För mottagning av meddelanden utvecklades  en gemensam mottagarfunktion som kom att bli standarden för alla enheter i systemet.
Att ha en gemensam mottagarfunktion underlättade vid implementering av nya periferienheter och minskade mängden kod.


\subsection{Utveckling av centralenheten}
Det första som utvecklades på centralenheten var DICP-funktionaliteten och andra grundläggande CAN-funktioner.
Centralenhetens logik för att upptäcka alarm utvecklades även.
Dessa funktioner prioriterades först för att kunna testa att central- och periferienhet kunde kommunicera med varandra.

Efter att grunden till kommunikationen lades utvecklades logiken på centralenheten för att hålla koll på alla periferienheter som var uppkopplade till systemet.
Denna informationen behövdes på centralenheten för implementationen av pollingsystemet.
Med den informationen kunde centralenheten veta till vilken periferienhetenhet nästa pollingmeddelande skulle skickas till och ifall ett pollingmeddelande inte besvarats av en periferienhet.
För utvecklingen av pollingsystemet hämtades inspiration ifrån TCP (Transmission Control Protocol), ett protokoll för kommunikation över internet.
Beslutet att modellera pollingsystemet efter TCP togs för att dess funktionalitet för att erkänna mottagen data lämpade sig väl.
Anledningen till att ett pollingsystem utvecklades var för att det var ett effektivt skydd mot återuppspelningsattacker och angrepp där periferienheter kopplas ifrån systemet.

Efter att pollingsystemet färdigställdes skapades funktioner för att jämföra inmatningen från en knappsats gentemot ett förutbestämt lösenord.
Med det implementerades logiken för att avlarma alla periferienheter vid upptäck inmatning av rätt lösenord.

I och med att periferienheterna expanderades till att innefatta flera sensorer expanderades även centralenheten till att kunna hantera flera sensorer på samma periferienhet.
Detta innebar att centralenheten uppdaterades till att även hålla koll på hur många sensorer som var uppkopplade till varje periferienhet och ifrån vilken sensor som ett larm uppstod ifrån.

För att användaren skulle kunna styra systemet skapades först funktioner för att kunna läsa tangentbordsinmatning via USART.
Därefter fastställdes en struktur för kommandon. Utifrån det implementerades det funktioner för att kunna tolka ifall det fanns ett kommando i en textsträng.
Funktioner som motsvarade varsin kommandotyp utvecklades sedan, till exempel funktionen \textit{set\_ndoors} som motsvarar \textit{ndoors} kommandot. 

%Centralenhetens uppgift var till en början att tilldela ID:n till periferienheterna och den agerade som centrum i kommunikationen.
%När alla enheter i systemet hade en unik identifikation kunde periferienheten skicka olika kommandon till de olika enheterna.
%Kommandon utvecklades för att konfigurera de olika enheterna.
%För att skydda systemet mot attacker implementerades ett lösenordsskydd i centralenhetens USART.

%Centralenhetens pollingsystem utvecklades i syfte att stoppa återuppspelningsattacker och att upptäcka eventuella bortfall av periferienheter.
%Pollingsystemet började som en idé på papper där teoretiska meddelanden ritades ut och manuellt stegades fram.
%Innehållet i meddelandet tog stor inspiration från datorkommunikationens TCP (Transfer Control Protocol).
%TCP är ett protokoll där avsändaren kan garantera att all data når mottagaren genom bekräftelser.

\subsection{Utveckling av periferienheterna}
Utvecklingen av periferienheterna startade med att utveckla funktioner för att läsa av sensorvärden.
När periferienheterna fick förmågan att läsa av sina sensorer upptäcktes det att det inte fanns något sätt att förmedla detta till utvecklaren.
För att lösa detta utvecklades ett antal funktioner som gjorde det möjligt att skriva ut olika typer av data till USART.

För identifiering av periferienheterna krävs en identifikation.
Tilldelningen av identifikationen sker automatiskt vid uppkoppling av periferienheten med hjälp av DICP.
DICP är ett protokoll baserat på DHCP (Dynamic Host Controller Protocol).
Protokollet valdes eftersom dess funktionalitet var liknade den funktionalitet som söktes.
DHCP tilldelar automatiskt IP(Internet Protocol)-adresser på ett nätverk.

Tilldelningen sker via att skicka ett meddelande på en generisk adress så kallad \textit{Broadcast Address} på engelska.
Den generiska adressen valdes att bli 15 eftersom indexeringen av periferienheter började på adress noll och att 15 är det största möjliga ID-värdet.
DICP använder sig av den generiska adressen för att anropa centralenheten i en ID-förfrågan.
Centralenheten svarar på ett DICP-meddelande med ett ledigt ID och periferienheten antar detta ID.

Med hjälp av identifikationer hade nu periferienheterna möjlighet att skicka meddelanden.
Meddelanden av typen ALARM utvecklades och kod för att väcka dessa meddelanden skrevs.
Tillsammans med funktionen att väcka ett ALARM utvecklades funktionalitet för att till exempel avlarma och ändra antalet underenheter.

Under projektet gång upptäcktes en felaktighet i en av periferienheterna.
Dörrperiferienheten hade endast stöd för en dörr och saknade funktionalitet för lampor.
Samtidigt hade rörelseperiferienheten endast stöd för avståndssensorn.
Detta ledde till att många typer av meddelande behövde skrivas om.
Hade specifikationen för periferienheterna studerats noggrannare hade detta kunnat undvikas.

\newpage
\section{Resultat}
Denna sektion syftar till att presentera det slutgiltiga resultatet av systemet samt att reflektera och diskutera kring utfallet utifrån de mål och syfte som ställts upp i rapportens inledning.
Sektionen inleds med resultaten från systemet i sin helhet följt av djupare granskning av delsystemen.
Resultat från utförda tester redovisas i denna sektion.

\subsection{Larmsystemet}
%Systemet utför sin grundligaste uppgift
%Systemets uppgift är att meddela användaren om avvikelse uppstår vid någon av sensorerna.
En sensor skickar vid upptäckt intrång en signal till centralenheten som i sin tur larmar och på så vis uppmärksammar användaren om intrång.
Möjlighet finns för användaren att ansluta just de sensorer han eller hon behöver.
Att sedan ställa in känsligheten hos sensorerna och på vilka grunder periferienheterna ska larma är implementerat och fungerar som väntat.

Systemet stödjer i teorin att upp till 15 enheter ansluts. Detta är inget som har kunnat verifieras då tillgång till den mängd hårdvara ej finns.
Som man kan se i testet ``Systemets prestanda med ansluten störenhet'' i bilaga \ref{A:störenhet} vidhåller systemet däremot sin funktionalitet när en störningsenhet är ansluten, detta tyder på att systemet kan hantera 15 anslutna enheter.

\subsubsection{Centralenhet}
Centralenheten fungerar som förväntat.
Vid larmutbrott skickas ett meddelande till en ansluten terminal via centralenheten.
I terminalen skrivs detaljer om larmet ut.
Larmet bryts via knappsatsen som är kopplad till centralenheten.
Utöver att centralenheten larmar via alarmmeddelande larmar även centralenheten vid tappad kontakt av periferienheterna och upptäckt felaktig data i ett pollingsvar.

Centralenheten lagrar det antal enheter som är anslutna till systemet och skickar kontinuerligt förfrågningsmeddelanden till enheterna.
Förfrågningarna innehåller ett meddelande som periferienheten förväntas besvara vilket skyddar från återuppspelningsattacker.

Gränssnittet som tillåter användaren att göra inställningar i systemet fungerar men är minimalt.
Gränssnittet är effektivt då endast ett fåtal knapptryckningar behövs för att justera en inställning.
Däremot är det lätt att göra misstag och precis vad varje kommando gör kan upplevas som otydligt.
Att expandera gränssnittet skulle göra systemet mer användarvänligt och enklare att hantera.

\subsubsection{Dörrperiferienhet}
Enheten kan hålla upp till sju dörrar under bevakning.
Varje dörr är kopplad till en grön ljusdiod och varje enhet är kopplad till en röd ljusdiod.
15 av 16 GPIOE-portpinnar används, sju dörrsensorer, sju gröna ljusdioder, en röd ljusdiod.
Sju dörrar per periferienhet valdes för att reservera resterande GPIO-portar.

Konstruktionen av enheten möjliggör för både ett lokalt alarm och även ett alarm som skickas till centralenheten.
Primärt kommer periferienheterna alarmera lokalt.
Om dörrlarmet stått längre än ett bestämt tidsintervall kommer även centralenheten alarmera.
I centralenheten kan tidsintervaller konfigureras.

\subsubsection{Rörelseperiferienhet}
Rörelseperiferienheten kombinerar två olika typer av sensorer, avstånds- och vibrationssensorer.
Avståndssensorn har möjligheten att upptäcka om något rör sig utan att det gör kontakt.
Vid kontakt har vibrationssensorn möjlighet att alarmera.

\begin{description}
  \item[Avståndsensor] Avståndssensorn observerar det senaste avståndet.
  Om avståndet underskrider två gånger toleransen alarmerar periferienheten lokalt.
  Däremot, om avståndet är mindre än toleransen skickas ett alarm till centralenheten.
  Toleransen går att bestämma på centralenheten.

  Avståndsintervallet för avståndssensorn är 2 cm till 360 cm. Ett prestandatest gjordes på detta och går att finna under bilaga \ref{A:avstand}.

  \item[Vibrationssensor] Vibrationssensorn läses av kontinuerligt.
  Om vibrationssensor upptäcker en vibration skickar den ett alarm till centralenheten omedelbart.
  Känsligheten för sensorn går att justera på sensorn med hjälp av en skruvmejsel.

  \end{description}

\subsubsection{Kommunikationsprotokoll}
Kommunikationsprotokollet fungerar väl och är motståndskraftigt gentemot återuppspelningsattacker.
Protokollet är designat för att så få meddelanden som möjligt ska behöva skickas.
I och med det har funktionalitet så som meddelandebekräftelser inte implementerats.
Det är en funktionalitet som potentiellt skulle göra systemet stabilare och mer motståndskraftigt mot attacker.
Istället har ett robust pollingsystem byggts som gör systemet tillräckligt stabilt. %Test här som visar på stabilitet?

Det slutgiltiga kommunikationsprotokollet skiljer sig från den ursprungliga designen.
Tanken var att varje meddelandetyp skulle ha en egen struktur.
Då ett meddelande behöver skickas skulle en funktion anropas för att skapade ett meddelande av korrekt typ.
Meddelandetypen skulle sedan kodas ned till ett generellt CAN-meddelande som kunde läggas ut på bussen och skickas.
Denna design övergavs tidigt i utvecklingsstadiet då insikten kom att de unika meddelandestrukturerna kunde strykas.
Istället skapas ett generellt CAN-meddelande direkt då ovan nämnd funktion kallas.
Strukturen av det generella CAN-meddelandet, som kan ses i Tabell \ref{tab:meddelandestruktur}, håller tillräckligt med information för de meddelandetyper som används.
%\todo{Dem klagade på referensen}
%tror denna är lungt för vi refererar tidigare


\subsection{Extra funktionalitet}
I målsättningen presenterades viss funktionalitet som skulle implementeras utöver larmsystemets grundligaste funktioner.
I detta stycke presenteras resultaten av detta.

\subsubsection{Självlärande system för ID hantering}
Systemet fungerar enligt specifikationerna för självlärandet.
När en periferienhet ansluter sig till systemet skickar den en förfrågan om att bli tilldelad ett enhets-ID till centralenheten.
Detta meddelande innehåller även information om periferienheten för centralenhet.
Funktionaliteten kunde verifieras då centralenheten skrev till terminalen via USART att den mottagit en förfrågan för ett ID och periferienheten skrev att den tilldelades ett ID.

\subsubsection{Skydd mot återuppspelningsattacker}
Systemets skydd mot denna typ av attacker var effektivt.
Vid återuppspelning av den inspelade busstrafiken orsakades ett alarm i systemet på ett eller annat sätt.
Detta beror på att centralenheten konstant skickar ut nya pollingmeddelanden innehållande en ny nyckel baserad på systemtiden.
Periferienheterna inverterar nyckeln och skickar tillbaka den i deras svar på pollingmeddelandet.
Den återuppspelade trafiken kommer aldrig innehålla den senaste nyckeln och därmed kommer ett alarm sättas hos centralenheten när den upptäcker den felaktiga nyckeln.

Skulle en enhet kopplas ifrån CAN-bussen sätts ett alarm genast på centralenheten.
Detta gör att en angripare inte kan byta ut systemets periferienheter mot sina egna enheter.

\begin{comment}
\todo{Fake news!!!}
\subsubsection{Testläge}
Möjlighet finns att i koden för centralenheten aktivera ett så kallat testläge.
Då centralenheten körs i detta testläge skrivs information om de meddelanden som tas emot och skickas från centralen ut till ansluten konsol.
Detta underlättar vid felsökning och installation av systemet.

\todo{Förbättringsmöjligheter i slutsatser}
Det finns förbättringsmöjligheter för testläget i den slutgiltiga produkten.
Till exempel möjlighet att få felmeddelanden utskrivna.
Mer felsökningsverktyg är alltid uppskattat.
Inom ramarna för detta projekt fyller testläget däremot sin funktion.
\end{comment}

\subsubsection{Lokalt larm}
Från ansluten konsol finns möjlighet för användaren att konfigurera precis hur och när varje periferienhet ska alarmera.
Om användaren vill att ett lokalt larm ska aktiveras vid den enhet som larmar finns den möjligheten.
Vid larm tänds då en röd lampa på enheten som indikerar att ett lokalt larm har gått.
Efter en tid som specificeras av användaren startas även ett larm från centralenheten så vida det lokala larmet inte har avaktiverats av användaren.
Denna funktionalitet fungerar som väntat och kan enkelt aktiveras av användaren via ansluten konsol.

\subsection{Tester}
För att verifiera att varje del av systemet fungerade enligt specifikationen testades systemet utförligt.
En mall för genomförelse och dokumentation av tester utvecklades så att olika tester sköttes enhetligt.
Denna sektion refererar till tester i bilaga~\ref{bilaga-tester}, dock är dessa inte alla tester som har utförts.
Många tester har utförts under utvecklingen av systemet.
Dessa tester har hjälpt identifiera problem så att de har kunnat åtgärdas innan nästa funktion utvecklades.
Testerna som har utförts visar att systemet fungerar enligt projektets specifikationer.
De visar även att systemet är robust då det fungerar även vid störningar och är skyddat mot enkla attacker.

%Mall för tester.
\begin{comment}
\subsubsection{Sak som testas}
\begin{description}
\item[Komponent] Den del av systemet som ska testas.

\item[Testsyfte] Vilken funktionalitet som testas.

\item[Utförande] Hur testet ska utföras och vilka specifika fall som testas.

\item[Resultat] Resultatet av testet.

\item[Analys] Vad testresultatet innebär; om komponenten fungerar som planerat, behöver testas ytterligare eller om vidare utveckling krävs.
\end{description}
\end{comment}

%Systemet fungerar som önskat där enhetens alla delar kommunicerar och sammarbetar vilket resulterar i ett effektivt larmsystem.

\section{Slutsatser}
Detta kapitel innehåller en reflektion på den slutgiltiga produkten och diskussion kring projektets drift.

\subsection{Resultat i förhållande till mål}

Larmsystemet uppfyller nästan alla de mål som ställdes i början av arbetet.
Centralenheten kan enligt specifikationerna konfigurera periferienheterna och ställa larm när ett alarm kommer ifrån en periferienheten.
Dessutom utvecklades ytterligare kommandon hos centralenheten utöver de mål som specificerades.

Periferienheterna fungerar enligt de krav som ställts och har även ytterligare funktionalitet.
Det går exempelvis att på rörelseperiferienheten att stänga av sensorerna vilket inte ursprungligen var planerat. 

Det ända målet som inte uppfylldes var testläget. 
Testläget prioriterades inte då kommunikation över CAN-bussen etablerades mycket tidigt. 
Det fanns ingen anledning för att bygga ett system för att emulera CAN-kommunikation när faktisk CAN-kommunikation var på plats.

Anledningen till att slutprodukten uppfyller närapå alla mål och även expanderats ytterligare beror till stor del på hur arbetsprocessen delades upp.
Att arbetet delades upp i tre delar i början gjorde att grundläggande funktionalitet snabbt utvecklades samtidigt inom de tre områdena.
Detta gjorde att central- och periferienheterna tidigt kunde kommunicera med varandra enligt kommunikationsprotokollet över CAN-bussen vilket förenklade processen att testa systemet.
På grund av detta gick det fort att verifiera att kod fungerade korrekt och gå vidare till att skriva ny kod.

\subsection{Användningsområden}
Det främsta användningsområdet för larmsystemet är att övervaka mindre lokaler genom att placera sensorer runtom utrymmet.
Systemet kan till exempel installeras i ett kontor.
Där skulle dörrar, fönster samt värdesaker så som datorer och skärmar kunna utrustas med sensorer.
Systemets användare skulle i så fall ha ett gott skydd mot stöld och oönskade intrång.

Larmsystemets olika sensorer tillsammans med det inbyggda skyddet mot attacker gör att det finns få möjligheter för att göra inbrott utan att avlösa ett alarm.
Den möjligheten som finns är om inbrottstjuvarna besitter kunskap om systemet och har fysisk åtkomst till CAN-bussen.

%För att förebygga inbrott kan även informationen att det finns ett larmsystem i bostaden framföras med skyltar och det i sig förhindrar de allra flesta inbrottsförsök.
Till skillnad från vad många tänker så hindrar larm inte de flesta inbrottstjuvar, men de minskar konsekvenserna av ett inbrott[2]. Larm sätter en tidspress på inbrottstjuvar vilket gör att de inte hinner begå ett fullständigt och färre saker blir stulna samt att skadorna blir mindre.

%\todo{Källa. Förbättra}
\subsection{Förbättringsmöjligheter}
Det finns många utvecklingsmöjligheter för systemet.
Till exempel är systemet känsligt mot attackerare som känner till hur systemet fungerar.
Detta skulle kunna förebyggas med ett mer avancerat pollingsystem med krypterade nycklar.
Möjlighet att utöka med fler typer av periferienheter finns också.
Centralenheten och kommunikationsprotokollet har goda förutsättningar för att expanderas för att kunna hantera fler typer av periferienheter.
Ett mer sofistikerat testläge skulle även kunna implementeras.
Detta skulle underlätta utvecklingen av nya delsystem och därmed förenkla vidareutveckling.

\subsection{Reflektion av projektdrift}
Det slutgiltiga systemet är ett bevis på att det är möjligt att konstruera ett enkelt larmsystem med grundliga kunskaper inom maskinorienterad programmering och MD407-kort.

De restriktioner som har legat på sociala sammankomster under utvecklingstiden har påverkat arbetet. Vid ett fåtal tillfällen har hela arbetsgruppen varit samlad i samma rum.
%Det visade sig däremot vara svårare än tänkt och deadlines för hårdvaran missades med några dagar.


\section{Referenser}
%\todo{Fler källor, dokumentation för sensorer. Kanske mer i inledning/slutsats}
\citewebpage{1}{BRA}{Anmälda brott}{2019}{https://www.bra.se/download/18.7d27ebd916ea64de5304d238/1585555596510/100La-2019.xls}{2020-09-20}

\citewebpage{2}{larm.se}{Har hemlarm någon effekt?}{}{https://larm.se/effekten-av-ett-inbrottslarm}{2020-10-29}

\citewebpage{3}{Elec Freaks}{Ultrasonic Ranging Module HC - SR04}{}{https://chalmers.instructure.com/courses/10285/files/folder/Kursmaterial\%20\%26\%20Dokument/H\%C3\%A5rdvarurelaterat?preview=662574}{2020-11-01}

\citewebpage{4}{e-Gizmo}{Vibration sensor module}{2017}{https://chalmers.instructure.com/courses/10285/files/folder/Kursmaterial\%20\%26\%20Dokument/H\%C3\%A5rdvarurelaterat?preview=662575}{2020-11-01}

\newpage
\appendix
\section{Tester}
\label{bilaga-tester}
\subsection{Funktionstester}
Här listas tester för systemets olika funktioner.
Dessa tester är för att se till så att systemet fungerar som planerat och för att hitta problem.

\subsubsection{CAN-test}
\begin{description}
\item[Komponent] Främst ska CAN testas, dock användes två enheter så att något kunde skicka och ta emot meddelanden.

\item[Testsyfte] Möjligheten att skicka och ta emot meddelanden.

\item[Utförande] Testet ska utföras så att en enhet skickar meddelanden till centralenheten som visar när den tar emot dem.

\item[Resultat] Centralenheten tog emot meddelanden skickade från periferienhet.

\item[Analys] Den grundläggande funktionen av att skicka CAN-meddelanden fungerar och vi kan gå vidare till mer avancerade operationer.
\end{description}

\subsubsection{Dörrlarmstest}
\begin{description}
\item[Komponent] Både centralenheten och dörrperiferienheten ska testas.

\item[Testsyfte] En full larmcykel skall testas; att dörrenheten kan skicka ett larm till centralenheten som tar emot det här meddelandet, och sen ska det larmas av.

\item[Utförande] Tre dörrar ska anslutas till dörrenheten. Larmet aktiveras från en dörr i taget och det larmas av däremellan genom att skriva in en fyr-siffrig kod till en knappsats ansluten till centralenheten.

\item[Resultat] Alla dörrar larmade till centralenheten då respektive dörr blev bortkopplad. Varje larm kunde återställas från centralenheten.

\item[Analys] Larmcykeln för dörrlarm med standardinställningar fungerar som förväntat.
Att avlarmning från centralenhet fungerar på ett larm från en dörrenhet tyder på att avlarmning även borde fungera på larm från andra enheter.
\end{description}

\subsubsection{Test av skydd mot återuppspelningsattacker}
\begin{description}
\item[Komponent] Centralenheten och dörrenheten.

\item[Testsyfte] Funktionaliteten som ska testas är att pollingsystemet mellan central- och periferienheter skyddar mot återuppspelningsattacker.

\item[Utförande] Hur testet ska utföras och vilka specifika fall som testas.
Alla enheter kopplas in och startas.
En enhet får spela in meddelanden mellan centralenheten och en dörrenhet.
%När den har spelat in en viss mängd trafik börjar den skicka ut den i samma sekvens som den tog emot den.
När den har spelat in en 32 meddelanden från CAN-bussen börjar den skicka ut dessa i samma sekvens som den tog emot dem.
Sedan testas också att dörrenheten kopplas ut, vilket i vanliga fall skulle ge upphov till ett larm och om systemet fungerar som planerat kommer det larmas även i detta fall.

\item[Resultat] Resultatet av testet. Vid många tillfällen så svarar återuppspelningsenheten inte som centralenheten förväntar sig.
Då något uppenbart är fel larmar centralenheten.

\item[Analys] Pollingsystemet fyller sitt syfte; systemet är inte känsligt mot återuppspelningsattacker.
\end{description}

\subsubsection{Test av hela systemet}
\begin{description}
\item[Komponent] Centralenheten, en dörrperiferienhet och en rörelseperiferienhet.

\item[Testsyfte] Hela systemet ska testas, det vill säga att larm från dörrar, vibrations- och avståndssensorn ska testas.
Det ska även testas om larmet aktiveras om en periferienhet tas bort.
Man ska kunna larma av med en knappsats och skicka kommandon till centralenheten via en ansluten dator.

\item[Utförande] Hur testet ska utföras och vilka specifika fall som testas.
Först testas dörrlarmet genom att koppla bort en sladd som simulerar en dörr.
Sedan aktiveras avståndssensorn genom att rörelse sker framför den och efter det testas vibrationssensorn genom att den blir rörd.
När dessa har testats så skall en periferienhet tas bort för att se om larmet aktiveras.
Sist testas kommandon från centralenheten, till exempel kommandot för att stänga av larmet på en dörr som sedan testas som innan, men nu förväntas inget larm.

\item[Resultat] Larm aktiveras när de är förväntade men inte när de inte borde aktiveras.
Kommandon från centralenheten gör det dem ska.

\item[Analys] Dessa resultat innebär att systemet i sin helhet fungerar och kan anses vara åtminstone mestadels klart.
\end{description}

\subsection{Prestandatest}
Här listas test på systemets prestanda.
Dessa test är till för att se hur bra systemet fungerar under press och för att se systemets gränser.

\subsubsection{Prestandatest av tid mellan larm på periferienhet och centralenhet}
\begin{description}
\item[Komponent] Både centralenheten och en valfri periferienhet ska testas.

\item[Testsyfte] Tiden som det tar att ett sensorvärde mäts till att ett alarm sätts på centralenheten.

\item[Utförande] Hela systemet ska kopplas upp för att testa systemet under normala omständigheter. Ett larm ska aktiveras på periferienheten och tiden som det tar tills att centralenheten larmar ska mätas.

\item[Resultat] Alarmet på centralenheten gick omedelbart. Det fanns inget syfte med att använda tidtagaruret.

\item[Analys] Under normala förhållanden går ett alarm omedelbart på centralenheten. Även under attack från en störenhet skulle tiden emellan knappt vara märkbar. Denna funktionalitet fungerar utmärkt
\end{description}

\subsubsection{Systemets prestanda med ansluten störenhet}
\label{A:störenhet}
\begin{description}
\item[Komponent] Centralenheten och en ansluten periferienhet ska testas, såväl som störenheten och CAN-bussen.
För testets syfte spelar det ingen roll vilken periferienhet som väljs, men i detta test valdes dörrenheten.

\item[Testsyfte] Det som ska testas är systemets förmåga att kunna skicka meddelanden mellan enheter när det finns många störningar på CAN-bussen.

\item[Utförande] Systemet kopplas upp och störenheten ansluts.
Därefter bevakas systemets vanliga funktioner för att se om det klarar påfrestningen.
Detta görs flera gånger med olika inställningar på störenheten.
Det som ställs in är hur ofta störenheten skickar meddelanden, alltså hur många störningar den lägger på CAN-bussen.

\item[Resultat] Systemet fungerar som vanligt utan problem med störenheten ansluten, även när den skickar meddelanden oavbrutet.

\item[Analys] Testet tyder på att systemet fungerar under press och att meddelandetyperna är prioriterade enligt specifikation.
\end{description}

\subsubsection{Avståndssensorns verkningsräckvidd}
\label{A:avstand}
\begin{description}
\item[Komponent] Räckvidden på avståndssensorn ska testas.

\item[Testsyfte] Räckvidden på sensorn testas i syfte av att kunna bestämma gränsvärden som avståndssensorn kan hantera.

\item[Utförande] Avståndssenorn kopplas upp och dess avstånd skrivs ut på USART.
Ett föremål placeras framför sensorn och vid förflyttning noteras det nya avståndet ner.
Både den undre och den övre gränsen testas på samma vis. 

\item[Resultat] Sensorn klarar att hantera avstånd från 2 cm upp till 360 cm innan den ger ifrån sig abnormaliteter.

\item[Analys] De värden som mättes upp i testet stämmer överens med dokumentationen av sensorn.
Det övre värdet kan variera på grund av faktorer som att rummet inte varit helt tomt och att föremål blockerat ljudvågorna från att komma fram.

\end{description}

\end{document}

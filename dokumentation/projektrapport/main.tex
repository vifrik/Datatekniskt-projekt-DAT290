\documentclass[a4paper]{article}

\usepackage[swedish]{babel}
\usepackage[T1]{fontenc}
\usepackage[utf8]{inputenc}
\usepackage{graphicx}
\usepackage{float}

\newcommand\namn{Larmsystem}

\restylefloat{table}

\begin{document}


\thispagestyle{empty}

\begin{center}
    \parskip=14pt
    \vspace*{3\parskip}

    {\LARGE Projektrapport DAT290}

    {\large \namn, grupp 11

    Titus Blosse, Viktor Frideen, Nazif Kadiroglu, Markus Moen, Lukas Schiavone, Fredrik Österström

    \today}

    \rule{7cm}{0.4pt}\\
\end{center}
\newpage

\thispagestyle{empty}

\tableofcontents
\newpage

\pagenumbering{arabic}

\section*{Ordlista}

\begin{description}
    \item[CAN:] Controller Area Network
    \item[IDE:] Integrated Development Environment
    \item[MD407:] Mikrodator av typen MD407
\end{description}
\newpage

\section{Inledning}
\subsection{Syfte}
Inbrott är ett vanligt problem i Sverige. År 2019 anmäldes 75.250 inbrottstölder. En minskning med 14\% från året innan \cite{brastold}. Vanliga sätt att förhindra inbrott är grannsamverkan, certifierade lås mot inbrott och hemlarm. Larmsystem minskar risken för ett inbrott och möjliggör ytterligare minskning både i Sverige och i resten av världen.

\subsection{Mål}

Produkten ska erbjuda ett flertal larmkomponenter som kan anslutas till en larmcentral och konfigureras för att passa kundens specifika situation. Systemet ska vara dokumenterat och fackmannamässigt utfört med förutsättningar för expansion.

Grunden i systemet kommer vara en centralenhet till vilken användaren kan ansluta periferienheter. Huvudsakligen kommer två periferienheter att utvecklas, ett dörrlarm och ett rörelselarm.

I mån av tid kommer extra funktionalitet implementeras. Denna extra funktionalitet är ett testläge för pereferienheterna, ett lokalt larm för rörelsesensorn, och ett inbyggt skydd mot ''replay attacker''.

\subsection{Bakgrund}

Beståndsdelarna i larmsystemet är en centralenhet och periferienheter. Komponenterna skapar tillsammans möjligheten att till exempel, larma på/av, koppla samman enheter eller inställning av befintliga enheter. Koppling sker via instruktioner som kan skickas både manuellt eller automatiskt via en av komponenterna.

\subsection{Arbetsmetod}

All mjukvara för projektet har skrivits i programspråket C. C är ett lågt programmeringsspråk vilket krävs för utveckling av mjukvara till mikrodatorn MD407, vilket är hårdvaran som utgör grunden för detta projekt. Utvecklingen av mjukvaran har skett i IDE:n (Integrated Development Environment) CodeLite. CodeLite användes då gruppmeddlemmarna sedan tidagare har erfarenhet med IDE:n samt att den lämpar sig bra för utveckling mot MD407.

Versionshantering har gjorts med hjälp av git. Ett gemensamt repository skapades på GitHub där all kod för projektet samlades.

Majoriteten av testningen utfördes direkt på hårdvaran i Chalmers lokaler. Viss testning har genomförts med hjälp av mjukvaran Simserver som tillåter att simulera MD407 kortet. Funktionaliteten hos Simserver är dock begränsad därför har den största delen av testningen behövts utföras i Chalmers laborationssalar.

Arbetsgruppen delades inledningsvis upp i tre undergrupper vilka ansvarade över olika delar av utvecklingen. Därefter utvecklade undergrupperna grunden för deras specifika område för att sedan i små steg utöka funktionaliteten. Under utvecklingens gång suddades gränserna mellan grupperna ut då dom separata mjukvarorna integrerades och gemensamt arbete blev fördelaktigt.


\section{Teknisk Beskrivning}
I detta avsnitt ges en fördjupad inblick i hur systemet som utvecklats under projektet fungerar och hänger ihop. Den mjuk- och hård-vara som projektet baserats på beskrivs även i mer detalj.

\subsection{Teknisk Bakgrund}
Hårdvaran vilket system körs på är en MD407 mikrodator. All mjukvara är skriven i programspårket C. MD407 har begränsat med minne vilket gör att programmen som skrivs för den måste vara små och effektiva. C är ett lågt programspråk vilket medför att programmeraren har stor kontroll över minneshanteringen i programmet. C lämpar sig alltså väl för utveckling mot MD407.

Systemet kretsar kring att flera MD407 sammankopplas, en MD407 agerar då som centralenhet och resterande används för att ansluta periferienheter så som dörrlarm och rörelsesensorer. MD407-korten kommunicerar via ett protokoll som baserats på CAN (Controllerar Area Network). CAN är ett färdigutvecklat protokoll för kommunikation mellan datorer. Ett kommunikationsprotokoll har sedan byggts uppepå CAN som är anpassat för systemet.

\subsection{Systemöversikt}

Vid uppstart väljs en MD407 som centralenhet för systemet. Från denna enhet körs huvudprogrammet. Centralenheten sköter den stora delen av systemets funktion. Centralenheten registrerar och håller koll på vilka periferienheter som är ansluta, kontrollerar kontinuerligt så att inget larm gått hos periferienheterna samt tar input från ansluten dator för att styra systemet.

Två olika periferienheter finns tillgängliga att ansluta till systemet. Ett dörrlarm, vilket kontrollerar ifall en dörr är öppen eller stängd, samt en rörelsesensor som mäter avståndet till föremålet framför den och registrerar ifall detta förändras.

\subsection{Delsystem}

Nedan presenteras systemets olika delar i djupare detalj.

\subsubsection{centralenhet}

\subsubsection{Dörrsensor}

\subsubsection{Rörelsesensor}

\subsubsection{Kommunikationsprotokoll}

För kommunikation mellan dom anslutna enheterna har ett kommunikationsprotokoll utvecklats. Basen till kommunikationsprotokollet är att så kallade CAN-meddelanden skickas mellan enheterna. Varje CAN-meddelande byggs upp av fem fält:
\begin{description}
  \item[Riktning:] Detta fält syftar till hurvida meddelandet går till eller från centralenheten. Fältet sätts till ett om meddelandet går från centralenhet till periferienhet och noll om meddelandet går från periferienhet till centralenhet. Detta fält tillåter en minskning av antalet meddelandetyper då en generell meddelandetyp så som ALARM kan användas åt båda hållen.
  \item[Meddelande-ID:] För varje möjligt meddelande som kan skickas finns en meddelandetyp, till exempel: ALARM och ERROR. Meddedelandetyperna är alla ordnade efter prioritet.
  \item[Enhets-ID:] varje ansluten enhet tilldelas ett unikt enhets-ID, en binär siffra 0-15. Ett lägre värde på enhets-ID:t innebär högre prioritet, därför tilldelas alltid centralenheten ID noll. Efter centralen tilldelas enheterna ett ID som relaterar till ordningen dom anslöt sig i. Alltså den tredje enheten som ansluts får ID tre.
  \item[Längd:] Längd är en indikation på hur mycket data som skickas med i meddelandets datafält. Många meddelanden skickas med längd noll då mycket information om meddelandet kan utläsas från meddelande-ID och enhets-ID.
  \item[Data:] Här lagras den data som skickas med i meddelandet. Upp till 64 bitar av data kan lagras i ett meddelande.
\end{description}


\section{Resultat}

\section{Slutsatser}


\bibliographystyle{IEEEtran}

\bibliography{referenser.bib}

\end{document}

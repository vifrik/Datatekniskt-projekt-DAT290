\documentclass{article}
\usepackage[utf8]{inputenc}

\renewcommand{\contentsname}{Innehåll}

\begin{document}
\begin{center}
\thispagestyle{empty}
\parskip=14pt%
\vspace*{3\parskip}%

{\LARGE Projektplan DAT290}

{\large Larmsystem, grupp 11

Titus Blosse, Viktor Frideen, Nazif Kadiroglu, Markus Moen, Lukas Schiavone, Fredrik Österström

\today}


\rule{7cm}{0.4pt}\\
\end{center}
\newpage

\thispagestyle{empty}
\tableofcontents
\newpage

\pagenumbering{arabic} 

\section{Syfte}
Lorem ipsum dolor sit amet, consectetur adipiscing elit. Maecenas gravida lectus a blandit tincidunt. Donec aliquet tempor lacus in cursus. Nulla consequat eros nibh, placerat placerat odio tempus ac. Nam eget imperdiet urna, eget luctus erat. Nam leo sem, pretium vitae orci eu, malesuada congue felis. Integer efficitur arcu a quam aliquet vehicula. Donec ullamcorper facilisis fringilla. Duis vehicula augue eros, eget fringilla est finibus sed. Vivamus imperdiet elit quis purus convallis interdum. Vivamus consequat tortor in feugiat scelerisque. Sed tincidunt sodales lacus sed mattis.\par
Maecenas vel leo luctus, sagittis elit sed, pulvinar ipsum. Nulla rutrum consequat arcu. Nullam facilisis imperdiet urna, a fermentum erat aliquet vitae. Mauris faucibus iaculis orci, quis sagittis dolor cursus ut. Proin eu condimentum nisl. Proin pretium volutpat velit molestie ultricies. Ut sodales mauris eu magna maximus, vel consectetur enim pretium. Fusce vel mi feugiat, convallis massa non, suscipit risus. Phasellus gravida ligula eget sagittis faucibus. Curabitur suscipit turpis euismod vehicula elementum. Nulla vestibulum velit eu urna rutrum posuere. Curabitur augue purus, faucibus a diam quis, fermentum mattis mi. Etiam eget quam id lacus iaculis volutpat. Sed eget volutpat urna, quis porta est.

\section{Mål}

\section{Bakgrund}

\subsection{Referenser}

\subsection{Tekniska förutsättningar}

\section{Systemöversikt}

\section{Resursplan}

Gör en lista över gruppmedlemmarna och ange tydligt hur man kommunicerar
med var och en av dessa (till exempel, ange epostadresser).
Utgångspunkten är att alla gruppmedlemmar är tillgängliga genom hela
projektet. Om det finns undantag, ange dessa.

Ange vilka roller som de olika gruppmedlemmarna har. Det är lämpligt att
en person är ansvarig för ett distinkt område; delat ansvar är
komplicerat. Här är ett förslag på ansvarsområden inom gruppen. Andra
områden är möjliga, till exempel kan man dela upp ansvaret för olika
delar av projektet istället för olika (men var försiktig med det för det
kan leda till att gruppen blir splittrad). Notera att ansvaret inte
innebär att man gör mer av det praktiska, snarare att övervakar gruppens
arbete och uppmärksammar gruppen när något inte fungerar.

\begin{itemize}

\item Gruppledare: Ansvarar för kommunikation med kursens lärare, ansvarar
    för att kalla till och leda gruppmöten.

\item Administrativt dokumentationsansvarig: Ansvarar för att
    mötesprotokoll förs. Ansvarar för att gruppen i tid börjar på
    oppositionsrapport, planeringsrapport och så vidare och ansvarar för
    att dessa skickas in i tid.

\item Teknisk dokumentationsansvarig: Ser till att kod är väl kommenterad.
    Uppmärksammar gruppen ifall det uppstår problem med kod som inte är
    tillräckligt dokumenterad.

\item Kodstandardansvarig: Kollar igenom att all kod som utvecklas följer
    standarder för struktur, namn och så vidare som gruppen bestämt.
    Uppmärksammar gruppen ifall disciplinen behöver skärpas vad
    gäller kodkvalitet.

\item Testansvarig: Håller ordning på de olika test som behöver genomföras
    på mjukvara och hårdvara, ser till att tillräcklig tid ägnas åt
    testning och att den dokumenteras väl. Uppmärksammar gruppen ifall
    testningen inte utförs eller dokumenteras tillräckligt.

\item Resursansvarig: Ser till att hårdvaran finns tillgänglig när den
    behövs, ansvarar för att olika verktyg ni använder för
    kommunikation, versionshantering och så vidare fungerar och
    används korrekt. Uppmärksammar gruppen ifall verktygen används fel
    eller inte fungerar väl.

\item Planeringsansvarig: Håller reda på de olika del-mål gruppen arbetar mot, kommunicerar med de andra gruppmedlemmarna om hur arbetet med dessa fortskrider och ger en rapport om det på veckomöten. Uppmärksammar gruppen så tidigt som möjligt om något mål inte gör framsteg så att gruppen kan planera om ifall det behövs eller så att arbetsinsatser kan omfördelas.

    
\end{itemize}

Ange vilka lokaler som kan disponeras, och när. Ange vilken hårdvara och
vilken mjukvara som finns tillgänglig. I och med att en del hårdvara
görs tillgänglig på begäran, ange hur ni begär ut hårdvara.

Om gruppmedlemmar vill arbeta med projektet på distans, utanför
Chalmers, beskriv hur man lämpligen går tillväga.

\begin{description}
    \item[Titus Blosse, administrativt dokumentansvarig:] Ansvarar för att mötesprotokoll förs och att de olika rapporterna som ska skrivas under projektets gång såväl påbörjas som skickas in i tid.
    
    E-mail: titus.blosse@gmail.com
    
    \item[Viktor Frideen, planeringsansvarig:] Ansvarar för att informera gruppen om hur arbete med projektet och dess delmål fortgår och även för att uppmärksamma gruppen om de hamnar efter planeringen.
    
    E-mail: 
    
    \item[Nazif Kadiroglu, teknik dokumentansvarig:] Ansvarar för att all kod i projektet är väl dokumenterad.
    
    E-mail: 
    
    \item[Markus Moen, testansvarig:] Ansvarar för att tester på både hårdvara och mjukvara testas och dokumenteras väl.
    
    E-mail: markus.offersten@gmail.com
    
    \item[Lukas Schiavone, kodansvarig:] Ansvarar för att gruppen följer den kodstandard de har satt.
    
    E-mail: 
    
    \item[Fredrik Österström, gruppledare och resursansvarig:] Ansvarar för att kommunikation med kursens lärare, gruppmöten, hårdvarans tillgänglighet och att de verktyg gruppen har valt för kommunikation och versionhantering används väl.
    
    E-mail: 
\end{description}

\section{Milstolpar}

\section{Aktiviteter}

\section{Tidsplan}

\section{Mötesplan}

\section{Kommunikationsplan}

\section{Kvalitetsplan}

\section{Spelregler}


\end{document}

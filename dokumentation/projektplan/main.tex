\documentclass{article}
\usepackage[utf8]{inputenc}

\renewcommand{\contentsname}{Innehåll}

\begin{document}
\begin{center}
\thispagestyle{empty}
\parskip=14pt%
\vspace*{3\parskip}%

{\LARGE Projektplan DAT290}

{\large Larmsystem, grupp 11

Titus Blosse, Viktor Frideen, Nazif Kadiroglu, Markus Moen, Lukas Schiavone, Fredrik Österström

\today}


\rule{7cm}{0.4pt}\\
\end{center}
\newpage

\thispagestyle{empty}
\tableofcontents
\newpage

\pagenumbering{arabic} 

\section{Syfte}
Lorem ipsum dolor sit amet, consectetur adipiscing elit. Maecenas gravida lectus a blandit tincidunt. Donec aliquet tempor lacus in cursus. Nulla consequat eros nibh, placerat placerat odio tempus ac. Nam eget imperdiet urna, eget luctus erat. Nam leo sem, pretium vitae orci eu, malesuada congue felis. Integer efficitur arcu a quam aliquet vehicula. Donec ullamcorper facilisis fringilla. Duis vehicula augue eros, eget fringilla est finibus sed. Vivamus imperdiet elit quis purus convallis interdum. Vivamus consequat tortor in feugiat scelerisque. Sed tincidunt sodales lacus sed mattis.\par
Maecenas vel leo luctus, sagittis elit sed, pulvinar ipsum. Nulla rutrum consequat arcu. Nullam facilisis imperdiet urna, a fermentum erat aliquet vitae. Mauris faucibus iaculis orci, quis sagittis dolor cursus ut. Proin eu condimentum nisl. Proin pretium volutpat velit molestie ultricies. Ut sodales mauris eu magna maximus, vel consectetur enim pretium. Fusce vel mi feugiat, convallis massa non, suscipit risus. Phasellus gravida ligula eget sagittis faucibus. Curabitur suscipit turpis euismod vehicula elementum. Nulla vestibulum velit eu urna rutrum posuere. Curabitur augue purus, faucibus a diam quis, fermentum mattis mi. Etiam eget quam id lacus iaculis volutpat. Sed eget volutpat urna, quis porta est.

\section{Mål}

\section{Bakgrund}

\subsection{Referenser}

\subsection{Tekniska förutsättningar}

\section{Systemöversikt}

\section{Resursplan}

I följande lista finns e-mailadresser till gruppens medlemmar, men för intern kommunikation används både Discord och Messenger. Gruppmöten kommer i första hand a ske igenom Zoom.

Ansvar i följande lista syftar inte till att en ansvarig skall göra allt inom sitt ansvarsområde, utan till att den ansvarige ska se till att det sköts ordentligt utav hela gruppen.

\begin{description}
    \item[Titus Blosse, administrativt dokumentansvarig:] Ansvarar för att mötesprotokoll förs och att de olika rapporterna som ska skrivas under projektets gång såväl påbörjas som skickas in i tid.
    
    E-mail: titus.blosse@gmail.com
    
    \item[Viktor Frideen, planeringsansvarig:] Ansvarar för att informera gruppen om hur arbete med projektet och dess delmål fortgår och även för att uppmärksamma gruppen om de hamnar efter planeringen.
    
    E-mail: 
    
    \item[Nazif Kadiroglu, teknik dokumentansvarig:] Ansvarar för att all kod i projektet är väl dokumenterad.
    
    E-mail: 
    
    \item[Markus Moen, testansvarig:] Ansvarar för att tester på både hårdvara och mjukvara testas och dokumenteras väl.
    
    E-mail: markus.offersten@gmail.com
    
    \item[Lukas Schiavone, kodansvarig:] Ansvarar för att gruppen följer den kodstandard de har satt.
    
    E-mail: 
    
    \item[Fredrik Österström, gruppledare och resursansvarig:] Ansvarar för att kommunikation med kursens lärare, gruppmöten, hårdvarans tillgänglighet och att de verktyg gruppen har valt för kommunikation och versionhantering används väl.
    
    E-mail: 
\end{description}

Hårdvaran för detta projekt finns tillgänglig i rum 4209 i EDIT-huset på Chalmers campus. Detta rum, och med det hårdvaran, kan bokas via en anslagstavla som även den finns i EDIT-huset. Det är dock möjligt att anslagstavlan kommer bytas ut till ett digitalt alternativ. Hårdvaran som finns tillgänglig är: 

\begin{itemize}
    \item 3x MD407 kort
    \item 1x Avståndsmätare (ultraljud), HC-SR04
    \item 1x Vibrationssensor, "Flying-Fish" SW-18010P
    \item 1x Keypad
    \item 1x 7-segmentsdisplay
    \item 2x 4-polig RJ-11 kabel (används för CAN-bussen)
    \item 1x RJ-11 förgrening
    \item 2x Tiopolig flatkabel
    \item 3x USB-kabel
    \item 1x Kopplingsplatta
\end{itemize}

Mjukvaran som kommer användas är CodeLite vilket är fördelaktigt då den kan simulera MD407 korten och då gruppmedlemmarna har erfarenthet med denna mjukvara. GitHub används för versionshantering.

Mycket av arbetet kan ske på distans då endast test på hårdvaran kräver att gruppmedlemmar är på plats i Chalmers. För arbete på distans kan gruppen kommunicera via Discord, som stödjer både röst- och textbaserad kommunikation.

\section{Milstolpar}

\section{Aktiviteter}

\section{Tidsplan}

\section{Mötesplan}

\section{Kommunikationsplan}

\section{Kvalitetsplan}

\section{Spelregler}


\end{document}

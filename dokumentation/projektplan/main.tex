\documentclass{article}
\usepackage[swedish]{babel}
\usepackage[T1]{fontenc}
\usepackage[utf8]{inputenc}
\usepackage{graphicx}

\begin{document}
\begin{center}
\thispagestyle{empty}
\parskip=14pt%
\vspace*{3\parskip}%

{\LARGE Projektplan DAT290}

{\large Larmsystem, grupp 11

Titus Blosse, Viktor Frideen, Nazif Kadiroglu, Markus Moen, Lukas Schiavone, Fredrik Österström

\today}


\rule{7cm}{0.4pt}\\
\end{center}
\newpage

\thispagestyle{empty}
\tableofcontents
\newpage

\pagenumbering{arabic}

\section{Syfte}

År 2019 anmäldes 75250  inbrottsstölder i Sverige vilket är en minskning med 14\%\ från året innan \ref{brastold}. Detta är en stark minskning och en trend som vi gärna ser fortsätta gå åt samma håll. Därför har vi valt att utveckla ett larm som enkelt kan installeras i alla typer av bostäder och som ger ett grundligt skydd mot oönskat intrång. Vi eftersträvar att i den slutgiltiga produkten kunna erbjuda en rad olika larmkomponenter som enkelt kan anslutas till en larmcentral och konfigureras för att passa kundens specifika sitution. 

%75.250 inbrottsstölder 2019 \ref{brastold}

\section{Mål}

Systemet ska ge användaren bättre kontroll över sina tillhörigheter. Systemet ska vara väldokumenterat och fackmannamässigt utfört med goda förutsättningar för expansion.

\section{Bakgrund}

\subsection{Referenser}

\subsection{Tekniska förutsättningar}

\section{Systemöversikt}

\section{Resursplan}

I följande lista finns e-mailadresser till gruppens medlemmar, men för intern kommunikation används både Discord och Messenger. Gruppmöten kommer i första hand a ske igenom Zoom.

Ansvar i följande lista syftar inte till att en ansvarig skall göra allt inom sitt ansvarsområde, utan till att den ansvarige ska se till att det sköts ordentligt utav hela gruppen.

\begin{description}
    \item[Titus Blosse, administrativt dokumentansvarig:] Ansvarar för att mötesprotokoll förs och att de olika rapporterna som ska skrivas under projektets gång såväl påbörjas som skickas in i tid.

    E-mail: titus.blosse@gmail.com

    \item[Viktor Frideen, planeringsansvarig:] Ansvarar för att informera gruppen om hur arbete med projektet och dess delmål fortgår och även för att uppmärksamma gruppen om de hamnar efter planeringen.

    E-mail: viktor.frideen@outlook.com

    \item[Nazif Kadiroglu, teknik dokumentansvarig:] Ansvarar för att all kod i projektet är väl dokumenterad.

    E-mail:

    \item[Markus Moen, testansvarig:] Ansvarar för att tester på både hårdvara och mjukvara testas och dokumenteras väl.

    E-mail: markus.offersten@gmail.com

    \item[Lukas Schiavone, kodansvarig:] Ansvarar för att gruppen följer den kodstandard de har satt.

    E-mail:

    \item[Fredrik Österström, gruppledare och resursansvarig:] Ansvarar för att kommunikation med kursens lärare, gruppmöten, hårdvarans tillgänglighet och att de verktyg gruppen har valt för kommunikation och versionhantering används väl.

    E-mail:
\end{description}

Hårdvaran för detta projekt finns tillgänglig i rum 4209 i EDIT-huset på Chalmers campus. Detta rum, och med det hårdvaran, kan bokas via en anslagstavla som även den finns i EDIT-huset. Det är dock möjligt att anslagstavlan kommer bytas ut till ett digitalt alternativ. Hårdvaran som finns tillgänglig är:

\begin{itemize}
    \item 3x MD407 kort
    \item 1x Avståndsmätare (ultraljud), HC-SR04
    \item 1x Vibrationssensor, "Flying-Fish" SW-18010P
    \item 1x Keypad
    \item 1x 7-segmentsdisplay
    \item 2x 4-polig RJ-11 kabel (används för CAN-bussen)
    \item 1x RJ-11 förgrening
    \item 2x Tiopolig flatkabel
    \item 3x USB-kabel
    \item 1x Kopplingsplatta
\end{itemize}

Mjukvaran som kommer användas är CodeLite vilket är fördelaktigt då den kan simulera MD407 korten och då gruppmedlemmarna har erfarenthet med denna mjukvara. GitHub används för versionshantering.

Mycket av arbetet kan ske på distans då endast test på hårdvaran kräver att gruppmedlemmar är på plats i Chalmers. För arbete på distans kan gruppen kommunicera via Discord, som stödjer både röst- och textbaserad kommunikation.

\section{Milstolpar}

\begin{table}[!h]
    \begin{center}
    \begin{tabular}{ |c|c|c|c| }\hline
    Nr & Beskrivning & Datum \\\hline\hline
    %\multirow{3}{4em}{Multiple row} & cell2 & cell3 \\
    1 & Projektplan inlämnad & 2020-xx-xx \\\hline
    .. & ... & ... \\\hline
    .. & ... & ... \\\hline
    .. & ... & ... \\\hline
    \end{tabular}
    \caption{Milstolpar för projektet}
    \label{milstolpar}
    \end{center}
\end{table}

\section{Aktiviteter}

Gruppmedlemmarna förväntas spendera 200h vardera med ett totalt antal mantimmar motsvarande 1200h. De 200h som varje medlem i gruppen förväntas lägga ner innefattar all tid som spenderas under projektets gång.

\begin{table}[!h]
    \begin{center}
    \begin{tabular}{ |c|c|c| }\hline
    Beskrivning & Tidsåtgång \\\hline\hline
    Föreläsningar (1.5h/vecka, 8 veckor, 6 personer) & 72h \\\hline
    Projektmöten (2h/vecka, 8 veckor, 6 personer) & 96h \\\hline
    Projektledning (2h/vecka, 8 veckor, 2 personer) & 32h \\\hline
    Framtagning av LaTeX-mallar & 9h \\\hline
    Arbeta med projektplan & 100h \\\hline
    Första genomläsning av mirkodatordokumentation & 30h \\\hline
    Programmering av periferienhet, dörr & 30h \\\hline
    Programmering av periferienhet, rörelse & 30h \\\hline
    Programmering av centralenhet & 50h \\\hline
    Programmering av buss & 50h \\\hline
    ... & ... \\\hline
    Förbered/genomför demonstration & 40h \\\hline
    Arbeta med projektrapport & 300h \\\hline
    \end{tabular}
    \caption{Aktivitetslista för projektet}
    \label{milstolpar}
    \end{center}
\end{table}

\section{Tidsplan}

\section{Mötesplan}
\begin{table}[!h]
    \begin{center}
    \begin{tabular}{ |c|c|c| }
    \hline
    Datum & tid & Lokal \\
    \hline\hline
    02/09-20 & 13:00-15:00 & ED4209 \\\hline
    09/09-20 & 15:00-17:00 & Zoom \\\hline
    16/09-20 & 13:00-15:00 & Zoom \\\hline
    23/09-20 & 13:00-15:00 & Zoom \\\hline
    30/09-20 & 13:00-15:00 & Zoom \\\hline
    07/10-20 & 13:00-15:00 & Zoom \\\hline
    14/10-20 & 13:00-15:00 & Zoom \\\hline
    21/10-20 & 13:00-15:00 & Zoom \\\hline
    \end{tabular}
    \caption{Tabell över mötestillfällen}
    \label{motesplan}
    \end{center}
\end{table}

\section{Kommunikationsplan}

\section{Kvalitetsplan}

\section{Spelregler}

\bibliographystyle{ieeetr}
\bibliography{referenser}

\end{document}
